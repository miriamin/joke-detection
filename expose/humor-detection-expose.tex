\documentclass[12pt]{scrartcl}
\usepackage[utf8]{inputenc}
\usepackage[T1]{fontenc}
\usepackage[english]{babel}
\usepackage{amsmath}
\usepackage[round]{natbib} %More control for citations, adding BiBTeX compatibility.
 
\title{%
	Joke detection with neural networks \\
	\large Project Exposé}
\author{Miriam Amin}
\date{WS 2019/20}

\begin{document}

\maketitle
\tableofcontents

\section{Introduction}
Humor is a fundamental property of humans. Although scholars are analyzing and studying humor since the Ancient Times, it is until today not completely understood. In contrast to other NLP-related problems, the computational treatment of humor is far behind. 

Former research in computational humor was mainly carried out on Humor Generation and Humor Detection. As I showed in earlier work \citep{aminComputationalHumorAutomatic2019}, none of the humor generators presented so far were able to produce human-like humor. From my investigations I concluded two approaches which seemed promising for the advancement of joke generators -- a generative and a restrictive approach. A generative approach to humor generation would aim at exclusively producing humorous output by preselecting suitable topics to joke about. A restrictive approach on the other hand would consist of two systems: A system that produces texts with structural features of jokes and a second humor detection system that works as a filter letting only the humorous texts pass.  
One approach for such a filter would be a neural network for text classification with the target classes \texttt{joke} and \texttt{no joke}.

The aim of this project is to assess the feasability of current neural network architectures for text classification for the application as such a joke detector. In the following I will briefly present related work and earlier systems for joke detection. I will proceed by outlining the intended method and the data set that will be used for training the neural network. 

 
 \cite{aminComputationalHumorAutomatic2019}

\subsection{Related Work}
\subsection{Data Set}
The training data set consists of xxx labeled jokes and non-jokes. 

The xxx positive jokes examples originate from different websites, crawled by \cite{pungasDatasetEnglishPlaintext2017} and \cite{moudgilPythonScriptsBuilding2017}. 

Negative jokes were created with gpt-2 by using the positive jokes as training data. As I showed earlier \citep{aminComputationalHumorAutomatic2019}, gpt-2 is not able to produce humorous output. However, it can be used to use generate text that is similar to jokes. 

\subsection{}

\bibliographystyle{apalike}
\bibliography{CH_Vol2_Humor_Detection}

\end{document}


