\documentclass[12pt]{scrartcl}
\usepackage[utf8]{inputenc}
\usepackage[T1]{fontenc}
\usepackage[ngerman]{babel}
\usepackage{amsmath}
\usepackage[round]{natbib} %More control for citations, adding BiBTeX compatibility.
 
\title{Joke detection with neural networks}
\author{Miriam Amin}
\date{WS 2019/20}

\begin{document}

\maketitle
\tableofcontents

\section{Introduction}
Humor is a fundamental property of humans. Although scholars are analyzing and studying humor since the Ancient Times, it is until today not completely understood. In contrast to other NLP-related problems, the computational treatment of humor is far behind. 

Former research in computational humor was mainly carried out on Humor Generation and Humor Detection. As I showed in earlier work \citep{aminComputationalHumorAutomatic2019}, none of the humor generators presented so far were able to produce human-like humor. From my investigations I concluded two approaches which seemed promising for the advancement of joke generators -- a generative and a restrictive approach. A generative approach to humor generation would aim at exclusively producing humorous output by preselecting suitable topics to joke about. A restrictive approach on the other hand would consist of two systems: A system that produces texts with structural features of jokes and a second humor detection system that works as a filter letting only the humorous texts pass.  
One approach for such a filter would be a neural network for text classification with the target classes \texttt{joke} and \texttt{no joke}.

The 

 
 \cite{aminComputationalHumorAutomatic2019}

\subsection{Related Work}
\subsection{Data Set}
Take subset of my joke dataset as positive examples


Create jokes with gpt2-simple as negative examples that are similar to jokes but are not jokes
\subsection{}

\bibliographystyle{apalike}
\bibliography{CH_Vol2_Humor_Detection}

\end{document}


